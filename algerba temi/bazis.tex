\documentclass{article}

\usepackage{amsmath} 
\usepackage[utf8]{inputenc}
\usepackage[T2A]{fontenc}
\usepackage[bulgarian,english]{babel}
\usepackage{amssymb}
\usepackage{tcolorbox}

\newcommand*{\nullvec}{\vec{\mathcal{O}}}

\title{Базис, размерност координати}
\author{Александър Гуров}
\date{\datebulgarian{\today}}

\begin{document}
\maketitle
\begin{tcolorbox}[title = Определение 5.1, colframe = red!70!black, colback = red!10!white]
    Непразно множество $B$ на ненулево линейно пространство $V \neq \{\vec{\mathcal{O}} \}$ е
    базис на V, ако:\\
    (i) е линейно независима система вектори\\
    (ii) линейната обвивка $l=(B)=V$ на $B$ съвпада с V
\end{tcolorbox}
\begin{tcolorbox}[title = Пример 5.2, colframe = green!70!black, colback = green!10!white]
    Векторите
    \[
        e_i=(\underbrace{0,...,0}_\text{i-1},1,\underbrace{0,...,0}_\text{n-i}), \in F^n, \ \ 1\leq i\leq n
    \]
    с единствена нененулева компонента 1 на място i образуват базис на линейното пространство
    $F^n$ от наредените n-торки от поле $F$.
\end{tcolorbox}

\underline{Доказателство.} За доказателството ще използваме, че за произволни $x_1,...,x_n \in F$
е в сила:
\[
    x_1e_1+x_2e_2+...+x_ie_i+...+x_ne_n=(x_1,...,x_i,...,x_n)
\]
От $x_1e_1+x_2e_2+...+x_ie_i+...+x_ne_n=(x_1,...,x_i,...,x_n)=(0,...,0)\implies x_1=x_2=...=x_i=...=x_n=0$, следователно векторите $e_1,e_2,...e_n$ са линейно независими.
Произволната наредена n-торка $x=(x_1,..,x_n)\in F^n$ е линейна комбинация $x=e_1x_1+...+x_ne_n$
на $(e_1,...,e_n)\in F^n$ с коефициенти $(x_1,...,x_n)\in F$, следователно $l(e_1,...,e_n)=F^n$ и
по дефиниция е базис на $F^n$.
\begin{tcolorbox}[title = Определение 5.3, colframe = red!70!black, colback = red!10!white]
    Линейно пространство V над поле F е крайномерно, ако $V=\vec{\{\mathcal{O}\}}$ или V има краен
    базис $v_1,...,v_n$.
\end{tcolorbox}
\newpage
Линейното пространство на наредените n-торки $F^n$ над поле $F$ е \\
крайномерно, защото има краен базис:
\[
    e_1,...,e_n,\text{, където } e_i=(\underbrace{0,...,0}_\text{i-1},1,\underbrace{0,...,0}_\text{n-i}),
    \in F^n, \ \ 1\leq i\leq n
\]
\begin{tcolorbox}[title = Твърдение 5.4, colframe = blue!70!black, colback = blue!10!white]
    Линейно пространство V над поле F е крайномерно тогава и само тогава, когато линейната обвивка
    $l(a_1,...,a_n)=V$ е на краен брой вектори. В такъв случай, ако $V\neq\{\vec{\mathcal{O}}\}$,
    то можем да изберем базис на V, съставен от подмножество на $\{a_1,...,a_n\}$.
\end{tcolorbox}
\underline{Доказателство} Ако  $V=\{\vec{\mathcal{O}}\}$, то $V=l(\nullvec)$ и V има краен базис.
Ако $V$ има краен базис $e_1,...,e_n$ , то $l(e_1,...,e_n)=V$ е линейна обвивка на краен брой вектори
и $V$ e крайномерно пространство.
Ако $l(e_1,...,e_n)=V$ е линейна обвивка на краен брой вектори и
$e_i=\nullvec , \forall i: 1\leq i\leq n$, то $V=\{\nullvec\}$, $V=l(\nullvec)$ и V е крайномерно.
Ако $l(e_1,...,e_n)=V$ е линейна обвивка на краен брой вектори и съществува $e_i\neq\nullvec$,
където $1\leq i \leq n$, след преномериране на $e_i$ като $e_1$, то $e_1$ е линейно независим и
$l(e_1)\subsetneq V$. Ако не съществува $e_i\notin l(e_1)$ при $2\leq i\leq n$, то $V=l(e_1)$ и $e_1$
е краен базис на V, тоест V е крайномерно. В противен случай преномерираме $e_i\notin l(e_1)$ като $e_2$.
По \emph{Лемата на линейната алгебра} $e_1, e_2$ са линейно независими и образуват краен базис на $V=l(e_1,e_2)$,
следователно V e крайномерно. Тъй като $e_1,...e_n$ е краен базис, повтаряйки тази стъпка краен m на брой
пъти, като $1\leq m \leq n$, можем да съставим краен базис на V от линейно независими вектори $e_1,...,e_m$ с $l(e_1,...,e_m)=V$.

\begin{tcolorbox}[title = Твърдение 5.5, colframe = blue!70!black, colback = blue!10!white]
    Всеки два базиса на ненулево пространство V над поле F трябва да имат еднакъв брой независими вектори.
\end{tcolorbox}
\underline{Доказателство} Нека $a_1,...,a_n$ и $b_1,...,b_n$ са базиси на V. Тогава
\[
    a_1,...,a_n \in V = l(b_1,...,b_n)
\]
Спрямо \emph{Основната лема на линейна алгебра}, $n\leq m$, заради линейната независимост. Аналогично
\[
    b_1,...,b_n \in V = l(a_1,...,a_n)
\]
От тук следва, че $m\leq n$, спрямо \emph{Основната лема на линейна алгебра}. В този случай $m=n$ и
всички два базиса на линейно пространство имат еднакъв брой независими вектори.

\begin{tcolorbox}[title = Определение 5.6, colframe = red!70!black, colback = red!10!white]
    Броят на независими вектори на базис на линейно пространство V се нарича размерност на V и сe
    бележи с $dim(V)$. \\
    Размерността на нулево пространство $V=\{\nullvec\}: dim(V)=0$.\\
    Размерността на ненулево линейно пространство V, което не е крайномерно: $dim(V)=\infty$
\end{tcolorbox}

\begin{tcolorbox}[title = Твърдение 5.7, colframe = blue!70!black, colback = blue!10!white]
    Следните свойства са еквивалентни за векторите $e_1,...,e_n$ от линейно пространство V над поле F:\\
    (i) $e_1,...,e_n$ е базис на V\\
    (ii) всеки вектор $x\in V=l(e_1,...,e_n)$ има единствено представяне:
    \[
        x=x_1e_1+...+x_ne_n
    \]
    като линейна комбинация на векторите $e_1,...,e_n$ с коефициенти $x_1,...,x_n$ от полето F.
    Наредената n-торка $(x_1,...,x_n)$ представлява координатите на x спрямо базиса $e_1,...,e_n$.
\end{tcolorbox}

\underline{Доказателство} $(i)\Rightarrow(ii)$ Нека произволен вектор $x$, принадлежащ на
линейното пространство $V$, има две представяния като линейна комбинация на векторите
$e_1,...,e_n$ с коефициенти $x_1,...,x_n\in F$ и $y_1,...,y_n\in F$:
\[
    x_1e_1+...+x_ne_n=x=y_1e_1+...+y_ne_n
\]
\[
    (x_1-y_1)e_1+...+(x_n-y_n)e_n=\nullvec
\]
Съгласно линейната независимост на $e_1,...,e_n$ е изпълнено $x_i-y_i=0, \forall i\in[1,n]$,
следователно $x_i=y_i, \forall i\in[1,n]$ и двете представяния съвпадат, тоест x има едно единствено,
представяне.

$(i)\Rightarrow(ii)$ Щом x има единствено представяне като линейна комбинация на $e_1,...,e_n$,
то $x\in V = l(e_1,...,e_n)$. В случая когато $x=\nullvec$:
\[
    x_1e_1+...+x_ne_n=\nullvec=0e_1+...+0e_n
\]
съгласно Единствеността на представянето на нулевия вектор като линейна комбинация на $e_1,...,e_n$,
коефициентите  $x_1,...,x_n=0$.
\[
    0e_1+...+0e_n=\nullvec
\]
Следователно векторите $e_1,...,e_n$ са линейно независими, което ги прави базис на V.

\begin{tcolorbox}[title = Твърдение 5.8, colframe = blue!70!black, colback = blue!10!white]
    Нека $V$ е ненулево линейно пространство над поле $F$. В този случай е изпълнено:\\
    (i) $dim(V)=n$, тогава и само тогава, когато съществуват n на брой независими вектора $e_1,...,e_n$, принадлежащи на V, и всеки
    n+1 на брой произволни вектора $a_1,...,a_{n+1}$ са линейно зависими.\\
    (ii) $dim(V)=\infty$, тогава и само тогава, когато за всяко естествено число $n$ същестуват $n$ линейно независими вектора $e_1,...,e_n\in V$.
\end{tcolorbox}

\underline{Доказателство} (i) $dim(V)=n$, следователно съществуват n линейно независими вектора $e_1,...,e_n\in V$ са базис.
Нека вземем произволни $n+1$ вектора от  $a_1,...,a_{n+1}\in V$:
\[
    a_1,...,a_{n+1}\in V=l(e_1,...,e_n)
\]
Спрямо \emph{Основната емата на линейната алгебра}, $a_1,...,a_{n+1}$ са линейни зависими, защото $n+1>n$.
В обратната посока, нека $e_1,...,e_n$ са n линейно зависими вектора и $a_1,...,a_{n+1}\in V$ са n+1 линейно зависими вектора.
Нека допуснем, че $l(e_1,...,e_n)\neq V$, което означава, че $l(e_1,...,e_n)\subsetneq V$. Следователно
съществува $e_{n+1} \in V\setminus l(e_1,...,e_n) $, за който $e_1,...,e_n,e_{n+1}$ са линейно независими, но всички n+1 вектора са линейно зависими, и това противоречие доказва, че $l(e_1,...,e_n)=V$ и е базис на V.

(ii) Нека допуснем, че  $dim(V)=\infty$ и съществуват $n \geq 2$ линейно зависими вектора $e_1,...,e_n$.
Нека n е минималната възможна стойност, за която това е изпълнено. Следователно съществуват n-1 линейно зависими вектора $e_1,...,e_{n-1}$.
Съгласно (i) оттук следва, че $dim(V)=n-1$, което е противоречие и доказва дясната посока.
В обратната посока, нека $dim(V)\neq\infty$. От предположението знаем, че за естествено число n съществуват n линейно независими вектора $e_1,...,e_n\in V$.
Нека $dim(V)=n$. Тогава спрямо (i), трябва да съществуват n+1 линейно зависими вектора, което е в противоречие с предположението. Тогава $dim(V)=\infty$.

\begin{tcolorbox}[title = Твърдение 5.9, colframe = blue!70!black, colback = blue!10!white]
    Следните условия са еквивалентни за n вектора $e_1,...,e_n$ от n-мерно линейно пространство V над поле F:\\
    (i) $e_1,...,e_n$ са линейно независими\\
    (ii) $l(e_1,...,e_n)=V$
    (iii) $e_1,...,e_n$ е базис на V
\end{tcolorbox}

\underline{Доказателство} По определението на базис, от (iii) следват (i) и (ii).


$(i) \Rightarrow (ii)$ и $(iii)$ Твърдим, че ако $e_1,...,e_n$ са n линейно независими вектора от
n-мерно линейно пространство V, то $l(e_1,...e_n)=V$ и $e_1,...e_n$ е базис на V. Нека допуснем,
че $l(e_1,...e_n)\neq V$, тогава съществува $e_{n+1}\in V\setminus l(e_1,...e_n)$ и $e_1,...e_n,e_{n+1}$ са линейно независими по Лема за линейна независимост.
Но спрямо Твърдение 5.8 (i), щом $dim(V)=n$, n+1 вектора са линейно зависими. Това противоречие доказва, че ако $e_1,...e_n$ са линейно независими, то $l(e_1,...e_n)=V$.
$(ii) \Rightarrow (i)$ и $(iii)$ Твърдим, че ако $l(e_1,...e_n)=V$, то $e_1,...,e_n$ са линейно независими и следователно базис на V.
В противен случай, от линейната зависимост на $e_1,...,e_n$, следва съществуването на индекс $1\leq i\leq n$ с $a\in l(e_1,...,e_{n-1},e_{n+1},...,e_n)$. Тогава:
\[
    l(e_1,...,e_{n-1},e_i,e_{n+1},...,e_n)\subseteq l(e_1,...,e_{n-1},e_{n+1},...,e_n)
\]
и с включването
\[
    l(e_1,...,e_{n-1},e_{n+1},...,e_n)\subseteq l(e_1,...,e_{n-1},e_i,e_{n+1},...,e_n)
\]
получаваме
\[
    V=l(e_1,...,e_{n-1},e_{n+1},...,e_n)= l(e_1,...,e_n)
\]
Съгласно предишно доказателство на Твърдение 5.4, съществува базис на V, който се съдържа в множеството ${e_1,...,e_{n-1},e_{n+1},...,e_n}$
и размерността на V е $dim(V)\leq n-1$. Това е противоречие с предположението $dim(V)=n$ и доказва, че ако $l(e_1,...e_n)=V$, то $l(e_1,...e_n)=V$ и $e_1,...e_n$ е базис на V.
\begin{tcolorbox}[title = Твърдение 5.10, colframe = blue!70!black, colback = blue!10!white]
    Нека V е n-мерно линейно пространство над поле F, а W е подпространство на V. Тогава $dim(W)\leq dim(V)=n$ с равенство $dim(W)=dim(V)$, тогава и само тогава, когато $W=V$ съвпадат.
\end{tcolorbox}
\underline{Доказателство} Нека подпространството W на линейното пространство V има $dim(W)\geq dim(V)=n$. Тогава $dim(W)=m\in \mathbb{N}$, то съгласно предишно Твърдение 5.8(i), съществуват m линейно независими вектора $w_1,...,w_m\in W, \ m\geq n+1$.
В този случай $dim(W)=\infty$, по Твърдение 5.8 (ii) съществуват m линейно независими вектори $w_1,...,w_m\in W, \forall m\in\mathbb{N}$.
Независимо от дали W е крайномерно или не, имаме n+1 линейно независими вектора $w_1,...,w_n,w_{n+1}\in W \subseteq V$. Възоснова на Твърдение 5.8 (i),
това е противоречие на $dim(W)=n$ и доказва $dim(W)\leq dim(V)$. Ако $dim(W)=dim(V)=n$, то произволни n линейно независими вектора
\[
    e_1,...,e_n\in W\subseteq V
\]
и спрямо Твърдение 5.9 образуват базис на W и базис на V. Следователно
\[
    W=l(e_1,...,e_n)=V
\]
\begin{tcolorbox}[title = Твърдение 5.10, colframe = blue!70!black, colback = blue!10!white]
    Нека $b_1,...,b_k$ са линейно независими вектори от n-мерно линейно пространство V над поле F. Тогава $k \leq n$ и векторите
    $b_1,...,b_k$ могат да се допълнят до базис $b_1,...,b_k,b_{k+1}...,b_n$ на V.
\end{tcolorbox}
\underline{Доказателство} Нека $e_1,...,e_n$ е базис на V. Линейната независимост на
$b_1,...,b_k \in V = l(e_1,...,e_n)$ изисква $k \leq n$ съгласно \emph{Основната лема на линейната алгебрa}. Същото следва от Твърдение 5.8 (i) , съгласно което
произволни n + 1 вектора в n-мерно пространство V са линейно зависими, така че ако $b_1,..., b_k$ са линейно независими, то $k \leq n$.
Ако k = n, то линейно независимите вектори $b_1,..., b_n$ в n-мерно линейно пространство V образуват базис на V по Твърдение 5.9.
За k < n е в сила строго включване $l(b_1,..., b_k)\subsetneq V$, защото от $l(b_1,...,b_k) = V$ за линейно независими вектори $b_1,..., b_k$ следва, че $b_1,..., b_k$ е базис на V и
$n>dim(V ) = k$. Избираме вектор $b_{k+1} \in V \setminus l(b_1,..., b_k)$. Тогава $b_1,..., b_k, b_{k+1}$
са линейно независими по Лема за линейна независимост. Ако k + 1 = n, то $l(b_1,...,b_k,b_{k+1}) = V$ . В случая k + 1 < n имаме $l(b1, . . . , bk, bk+1)\subsetneq V$
и съществува $b_{k+2} \in V \setminus l(b1, . . . , bk, bk+1)$. Тогава векторите $b_1,...,b_k, b_{k+1}, b_{k+2}$ са линейно независими по Лема за линейна независимост.Ако продължим по
този начин, след краен брой стъпки получаваме n линейно независими вектора $b_1,...,b_k,b_{k+1},..., b_n$ от n-мерното пространство V така, че $b_1,...,b_k, b_{k+1},..., b_n$ е базис на V съгласно Твърдение 5.9.
\end{document}
