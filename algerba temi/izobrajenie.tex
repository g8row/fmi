\documentclass{article}

\usepackage{amsmath} 
\usepackage[utf8]{inputenc}
\usepackage[T2A]{fontenc}
\usepackage[bulgarian,english]{babel}
\usepackage{amssymb}
\usepackage{tcolorbox}

\newcommand*{\nullvec}{\vec{\mathcal{O}}}

\title{15. Матрица на линейно изображение на крайномерни пространства. Смяна на базис. 
Трансформация на матрицата на линейно изображение при смяна на базисите. Подобни матрици}
\author{Александър Гуров}
\date{\datebulgarian{\today}}
\newcommand{\dok}{\underline{Доказателство}}
\newcommand{\tvurdenie}[2]{
    \begin{tcolorbox}[title = #1 ,colframe = blue!70!black, colback = blue!10!white]
        #2
    \end{tcolorbox}
}
\newcommand{\opredelenie}[2]{
    \begin{tcolorbox}[title = #1 ,colframe = red!70!black, colback = red!10!white]
        #2
    \end{tcolorbox}
}
\newcommand{\primer}[2]{
    \begin{tcolorbox}[title = #1 ,colframe = blue!70!black, colback = blue!10!white]
        #2
    \end{tcolorbox}
}

\begin{document}
\maketitle

\tvurdenie{Лема 15.2 (\emph{Матричен запис на линейно изображение})}{
Нека $\varphi : U \rightarrow V$ е линейно изображение, $u=(u_1,...,u_m)$ е наредена m-торка, съставена от вектори $u_1,...,u_m\in U$, и
$A=(a_{i,j})_{i=1j=1}^{m \ \ n} \in M_{m\times n}(F)$. тогава
\[
    \varphi(uA)=\varphi(u)A
\]
за
\[
    uA:=(v_1,...,v_n), \ v_i=(u_1,...,u_m)\left(\begin{array}{c}
            a_{1j} \\
            a_{2j} \\
            \dots  \\
            a_{mj}
        \end{array}\right)=\sum_{i=1}^{m}a_{ij}u_i,
\]
\[
    \varphi(uA):=\varphi(v_1,...,v_n)=(\varphi(v_1),...,\varphi(v_m)) \text{ и } \varphi(u)=(\varphi(u_1),...,\varphi(u_m))
\]
}
\dok \ От равенството на j-тия стълб на $\varphi(uA)$ и j-тия стълб на $\varphi(u)A$:
\[
    \varphi(v_j)=\varphi\left(\sum_{i=1}^{m}a_{ij}u_i\right)=\sum_{i=1}^{m}a_{ij}\varphi(u_i)=(\varphi(u_1),..,\varphi(u_m))\left(\begin{array}{c}
            a_{1j} \\
            a_{2j} \\
            \dots  \\
            a_{mj}
        \end{array}\right)
\]
за всяко $1\leq j\leq n$.
\opredelenie{Определение 15.2}{
    Нека $\varphi: U\rightarrow V$ е линейно изображение, векторите $e_1,...,e_n$ образуват базис на
    линейното пространство U над поле F, векторите $f_1,...,f_n$ образуват базис на линейното
    пространство V над поле F. Матрицата:
    \[
        A=(\varphi(e_1),...,\varphi(e_n))\in M_{m\times n}(F)
    \]
}


\end{document}