\documentclass{article}

\usepackage{amsmath} 
\usepackage[utf8]{inputenc}
\usepackage[T2A]{fontenc}
\usepackage[bulgarian,english]{babel}
\usepackage{amssymb}
\usepackage{tcolorbox}

\newcommand*{\nullvec}{\vec{\mathcal{O}}}

\title{15. Матрица на линейно изображение на крайномерни пространства. Смяна на базис. 
Трансформация на матрицата на линейно изображение при смяна на базисите. Подобни матрици}
\author{Александър Гуров}
\date{\datebulgarian{\today}}

\newcommand{\dok}{\underline{Доказателство}\  }

\newcommand{\tvurdenie}[2]{
    \begin{tcolorbox}[title = #1 ,colframe = blue!70!black, colback = blue!10!white]
        #2
    \end{tcolorbox}
}
\newcommand{\opredelenie}[2]{
    \begin{tcolorbox}[title = #1 ,colframe = red!70!black, colback = red!10!white]
        #2
    \end{tcolorbox}
}
\newcommand{\primer}[2]{
    \begin{tcolorbox}[title = #1 ,colframe = blue!70!black, colback = blue!10!white]
        #2
    \end{tcolorbox}
}

\begin{document}
\maketitle

\tvurdenie{Лема 15.2 (\emph{Матричен запис на линейно изображение})}{
Нека $\varphi : U \rightarrow V$ е линейно изображение, $u=(u_1,...,u_m)$ е наредена m-торка, съставена от вектори $u_1,...,u_m\in U$, и
$A=(a_{i,j})_{i=1j=1}^{m \ \ n} \in M_{m\times n}(F)$. тогава
\[
    \varphi(uA)=\varphi(u)A
\]
за
\[
    uA:=(v_1,...,v_n), \ v_i=(u_1,...,u_m)\left(\begin{array}{c}
            a_{1j} \\
            a_{2j} \\
            \dots  \\
            a_{mj}
        \end{array}\right)=\sum_{i=1}^{m}a_{ij}u_i,
\]
\[
    \varphi(uA):=\varphi(v_1,...,v_n)=(\varphi(v_1),...,\varphi(v_m)) \text{ и } \varphi(u)=(\varphi(u_1),...,\varphi(u_m))
\]
}
\dokОт равенството на j-тия стълб на $\varphi(uA)$ и j-тия стълб на $\varphi(u)A$:
\[
    \varphi(v_j)=\varphi\left(\sum_{i=1}^{m}a_{ij}u_i\right)=\sum_{i=1}^{m}a_{ij}\varphi(u_i)=(\varphi(u_1),..,\varphi(u_m))\left(\begin{array}{c}
            a_{1j} \\
            a_{2j} \\
            \dots  \\
            a_{mj}
        \end{array}\right)
\]
за всяко $1\leq j\leq n$.
\opredelenie{Определение 15.2}{
Нека $\varphi: U\rightarrow V$ е линейно изображение, векторите $e_1,...,e_n$ образуват базис на
линейното пространство U над поле F, векторите $f_1,...,f_n$ образуват базис на линейното
пространство V над поле F. Матрицата:
\[
    A=(\varphi(e_1),...,\varphi(e_n))\in M_{m\times n}(F)
\]
образувана по стълбовете от координатите на векторите $\varphi(e_1),...,\varphi(e_n)\in V$ спрямо базиса $f=(f_1,...,f_n)$ на V
наричаме матрица на линейно изображение $\varphi$ спрямо базисите e и f. Еквивалентно:
\[
    \varphi(e)=fA
\]
за $\varphi(e):=(\varphi(e_1),...,\varphi(e_n) )$.
}

Матрицата A се описва еднозначно $\varphi$ чрез e и f, базиси на U и V, както чрез образите $\varphi(e_1),...,\varphi(e_n)$ на базиса e на U.

\tvurdenie{Лема 15.4  (\emph{Матрична форма на линейната независимост на вектори})}{
Нека $u_1,...,u_m$ са линейно независими вектори от линейно пространство U над поле F, $u = (u_1,..., u_m) \text{ и } A = (a_{ij})_{i=1j=1}^{m \ \ n}, B = (b_{ij})_{i=1j=1}^{m \ \ n} \in M_{m\times n}(F)$
са матрици с елементи от F. Тогава:\\
(i) от $uA = \underbrace{(\nullvec,...,\nullvec)}_{n}$ следва $A = \mathbb{O}_{m\times n}$;\\
(ii) от $uA = uB$ следва $A = B$.
}

\dok(i) За всяко $1 \leq j \leq n$, сравняването на j-тите компоненти на двете страни на
\[
    uA = \underbrace{(\nullvec,...,\nullvec)}_{n}
\]
дава
\[
    (u_1 \dots u_m)\left(\begin{array}{c}
            a_{1j} \\
            \dots  \\
            a_{mj}
        \end{array}\right)=\nullvec
\]
Но равенството
\[
    \sum_{s=1}^{m}a_{sj}u_s
\]
за линейно независимите вектори $u_1,..., u_m$ изисква $a_{sj} = 0$ за $\forall 1 \leq s \leq m$.
Това доказва $a_{sj} = 0$ за всички $1 \leq s \leq m, 1 \leq j \leq n$ и $A = \mathbb{O}_{m\times n}$.

(ii) Ако $uA = uB$, то от
\[
    u(A-B)=uA-uB=\underbrace{(\nullvec,...,\nullvec)}_{n}
\]
следва $A=B=\mathbb{O}_{m\times n}$\\ \\
Ако произволно $u\in U$ има координатите
\[
    x = \left(\begin{array}{c}
            x_1   \\
            \dots \\
            x_n
        \end{array}
    \right)\in M_{n\times 1}
\]
спрямо базиса $e=(e_1,...,e_n)$, то
\[
    u=\sum_{i-1}^{n}x_ie_i=(e_1 ... e_n)\left(\begin{array}{c}
            x_1   \\
            \dots \\
            x_n
        \end{array}
    \right)=ex
\]
Прилагайки $\varphi$ върху u:
\[
    \varphi(u)=\varphi(ex)=\varphi(e)x=(fA)x=f(Ax)
\]
съгласно Лема 15.1, Определение 15.2 за матрица на линейно изображение и
асоциативността на умножението на матрици. Ако
\[
    y = \left(\begin{array}{c}
            y_1   \\
            \dots \\
            y_m
        \end{array}
    \right)\in M_{m\times 1}
\]
са координатите на $\varphi(u)$ спрямо базиса $f = (f_1, . . . , f_m)$ на V , то $\varphi(u) = fy$,
откъдето
\[
    fy=\varphi(u)=f(Ax)
\]
\[
    y=Ax
\]
По този начин, за да пресметнем координатите y на образа $\varphi(u)$ на $u \varphi U$
относно базиса f на V трябва да умножим матрицата A на $\varphi$ спрямо e и f с
координатния стълб x на u спр ямо базиса e на U.\\
Например, нулевото линейно изображение $\mathbb{O} : U \rightarrow V , \mathbb{O}(u) = \nullvec_v , \forall u \in U$
на n-мерно пространство U в m-мерно пространство V има нулевата матрица
$\mathbb{O}_{m\times n} \in M_{m\times n}(F)$ спрямо произволен базис $e = (e_1, . . . , e_n)$ на U и произволен
базис $f = (f_1, . . . , f_m)$ на V . Причина за това е $\mathbb{O}(e_i) = 0f_1 + . . . + 0f_m$ за
произволно $1 \leq i \leq n$.

\opredelenie{Определение 15.4}{
    Ако $\varphi : U \rightarrow U$ е линеен оператор в n-мерно пространство U и $e = (e_1, . . . , e_n)$ е базис на U, то матрицата
    \[
        A = (\varphi(e_1), . . . , \varphi(e_n)) \in M_{m\times n}(F),
    \]
    образувана по стълбове от координатите на $\varphi(e_1), . . . , \varphi(e_n)$ спрямо
    $e_1, . . . , e_n$ се нарича матрица на $\varphi$ спрямо базиса e.
    Еквивалентно, A се определя от равенството
    \[
        \varphi(e) = eA.
    \]
}
\opredelenie{Определение 15.5}{
    Ако $e = (e_1, . . . , e_n)$ и $f = (f_1, . . . , f_n)$ са базиси на
    линейно пространство V над поле F, то матрицата
    \[
        T=(f_1 ... f_j ... f_n)=\left(\begin{array}{ccccc}
                t_{11} & ... & t_{1j} & ... & t_{1n} \\
                ...    & ... & ...    & ... & ...    \\
                t_{n1} & ... & t_{nj} & ... & t_{nn} \\
            \end{array}
        \right)\in M_{m\times n}(F)
    \]
    образувана по стълбовете на координатите на
    \[
        f_j=(e_1 \ ... \ e_n)\left(\begin{array}{c}
                t_{1j} \\
                ...    \\
                t_{nj} \\
            \end{array}
        \right)\in V, 1 \leq j \leq n
    \]
    спрямо базиса $e_1, . . . , e_n$ се нарича матрица на прехода от базиса
    $e = (e_1, . . . , e_n)$ към базиса $f = (f_1, . . . , f_n)$. Еквивалентно, матрицата на
    прехода $T \in M_{n\times n}(F)$ от базиса e към базиса f е единствената матрица, изпълняваща равенството
    \[
        f = (f_1, . . . , f_n) = (e_1, . . . , e_n)T = eT.
    \]
}
\tvurdenie{Твърдение 15.6}{
    Нека $e = (e_1, . . . , e_n)$ е базис на линейно пространство V над поле F, а $T \in M_{n\times n}(F)$ е квадратна матрица. В такъв случай,
    T е матрица на прехода от базиса e към базиса $f = (f_1, . . . , f_n) = eT$
    тогава и само тогава, когато матрицата T е неособена.
}
\dok Ако $f = eT$ е базис на V , то $e = fS$ за матрицата на
прехода $S \in M_{n\times n}(F)$ от базиса f към базиса e. Тогава
\[
    eE_n = e = fS = (eT)S = e(TS)
\]
откъдето $TS = E_n$ по Лема 15.3 (ii) за линейно независимите вектори $e_1, . . . , e_n$.
Следователно T е обратима, а оттам е и неособена матрица. Ако T е неособена матрица и $det(T)\neq 0$, то вектор-стълбовете на T са линейно
независими съгласно Твърдение 12.7, следователно векторите $f_1, . . . , f_n$, чиито координати спрямо базиса $e_1, . . . , e_n$ образуват вектор-стълбовете на T са
линейно независими и образуват базис на n-мерното линейно пространство V по Твърдение 5.12.

\tvurdenie{Твърдение 15.7}{
    Нека $e = (e1, . . . , en)$ и $f = (f1, . . . , fn)$ са базиси на
    линейно пространство V с матрица на прехода $T \in M_{n\times n}(F)$ от e към
    f. Тогава координатите $x \in M_{n\times 1}(F)$ на вектор $v \in V$ спрямо базиса
    e и координатите $y \in M_{n\times 1}(F)$ на същия вектор v спрямо базиса f са
    свързани с равенството
    \[
        x = T y.
    \]
}

\dok Съгласно $f = eT$ и $ex = v = fy$ имаме
\[
    ex = fy = (eT)y = e(Ty),
\]
откъдето $x = Ty$, съгласно 15.3 (ii) за линейно независимите вектори $e_1, . . . , e_n$.

\tvurdenie{Твърдение 15.8}{
    Нека $\varphi : U \rightarrow V$ е линейно изображение с матрица A
    спрямо базис $e = (e_1, . . . .e_n)$ на U и базис $f = (f_1, . . . , f_m)$ на V ,
    $e' = eT$ е друг базис на U с матрица на прехода T от $e$ към $e'$ и $f' = fS$ е друг
    базис на V с матрица на прехода S от $f$ към $f'$. Тогава матрицата на
    $\varphi$ спрямо базиса $e'$ на U и базиса $f'$ на V е
    \[
        B = S^{-1}AT.
    \]
}
\dok По Определение 15.2 за матрица на линейно изображение спрямо базисите на U и V имаме $\varphi(e) = fA$ и
$\varphi(e') = f'B$. Заместваме $e' = eT, f'= fS$ съгласно Определение 15.5 за матрица на прехода между
два базиса на линейно пространство. Като изпозваме Лема 15.1 и асоциативността на умножението на матрици, получаваме
\[
    f(AT) = (fA)T = \varphi(e)T = \varphi(eT) = \varphi(e') = f'B = (fS)B = f(SB).
\]

По Лема 15.3 (ii) за линейно независимите вектори $f_1, . . . , f_m$, това е достатъчно
за $AT = SB$. По Твърдение 15.6, матрицата на прехода S от базиса f на V към
базиса $f'$ на V e обратима и $B = S^{-1}AT$.

Ако $\varphi : U \rightarrow U$ е линеен оператор с матрица $A \in M_{n\times n}(F)$ спрямо
базис $e = (e_1, . . . , e_n)$ на U и $e' = eT$ е базис на U с матрица на прехода $T \in M_{n\times n}(F)$
от $e$ към $e'$, то матрицата на $\varphi$ спрямо базиса $e'$ е $B = T^{-1}AT$.

\opredelenie{Определение 15.9}{
    Квадратни матрици $A, B \in M{n\times n}(F)$ с един и същи размер са подобни, ако съществува обратима
    матрица $T \in M_{n\times n}(F)$, така че  $B = T^{-1}AT$.
}
\tvurdenie{Твърдение 15.10}{
    Квадратни матрици $A,B \in M{n\times n}(F)$ са подобни тогава и само тогава, когато съществува
    линеен оператор в n-мерно линейно пространство над F с матрици A и B спрямо подходящи базиси.
}
\dok От Твърдение 15.8 следва, че ако $\varphi : U \rightarrow U$ е линеен
оператор с матрица $A \in M_{n\times n}(F)$ спрямо произволен базис e на U, то матрицата
на $\varphi$ спрямо базиса $e' = eT$ с матрица на прехода $T \in M_{n\times n}(F)$ от $e$ към $e'$
е подобна на A и равна на B = $B = T^{-1}AT$.

Нека A и $B = T^{-1}AT$ са подобни матрици. Избираме базис $e = (e_1, . . . , e_n)$
на n-мерно линейно пространство U над F и разглеждаме линейния оператор
$\varphi : U \rightarrow U$ с матрица A спрямо базиса e. Матрицата T е неособена, така че
$e' = eT$ e базис на U съгласно Твърдение 15.6. По Твърдение 15.8, матрицата
на линейния оператор $\varphi$ спрямо базиса $e'$ на U е $T^{-1}AT=B$.
\end{document}