\documentclass{article}

\usepackage{amsmath}
\usepackage{amssymb}
\usepackage[utf8]{inputenc}
\usepackage[T2A]{fontenc}
\usepackage[bulgarian,english]{babel}
\usepackage{graphicx}
\graphicspath{ {./images/} }


\title{Домашно 2}
\author{Александър Гуров}
\date{\datebulgarian{\today}}

\begin{document}
\maketitle
\section*{Задача 1}

Ще дефинираме обикновен граф $G(V,E)$, като n на брой точките върху окръжността и точките
на пресичане ще представляват върховете, а свързващите прави и криви,
които са част от окръжността - ребрата.



Всяка пресечна точка се получава при пресичането на 2 прави. Следователно броя на пресечните точки
ще бъде равен на броя групи от 4 точки, лежащи на окръжността - $\binom{n}{4}$.
Броят на върховете на целия граф ще бъде сбора на броя на точките върху окръжността и на точките
на пресичане:
\[
    n+\binom{n}{4}
\]

Можем да намерим броя на правите, свързващи лежащите на окръжността n на брой точки - $\binom{n}{2}$.
Всеки връх, който е точка на пресичане разделя 2 прави на 4 ребра. Следователно броят
на ребрата ще бъде:
\[
    \binom{n}{2}+2\binom{n}{4}+n
\]

За решението на тази задача ще използваме Ойлеровата характеристика на многостените:
\[
    n-m+f=2
\]
\[
    f=2-n-\binom{n}{4}+\binom{n}{2}+2\binom{n}{4}+n
\]
\[
    f=1\binom{n}{4}+\binom{n}{2}+2
\]
Сега трябва да извадим 1 от получения резултат, защото не броим лицето на кръга.
Финалния отговор за броя лица е:
\[
    f=1\binom{n}{4}+\binom{n}{2}+1
\]

\section*{Задача 2}
Щом говорим за всеки обикновен граф, най-големият брой цветове нужни да оцветим граф ще бъде при пълен граф, а именно
\[
    \chi(K_n)=n
\]
Следователно:
\[
    \chi(G)\leq n
\]
Също така можем да получим това неравенство от формулите:
\[
    \begin{array}{c}
        \chi (G) \leq \Delta (G)+1 \\
        \Delta(G) \leq n-1
    \end{array}
    \implies
    \chi(G)\leq n
\]
Знаем броя ребра на пълен граф:
\[
    |E(K_n)|=\binom{n}{2}=\frac{n(n-1)}{2}
\]
Следователно:
\[
    |E(G)|=m\leq \binom{n}{2}
\]
\[
    m\leq \frac{n(n-1)}{2}
\]
Нека проверим твърдението в условието $\chi (G)\leq \frac{1}{2} + \sqrt{2m+\frac{1}{4}}$ като заместим $m$:
\[
    \chi (G)\leq \frac{1}{2} + \sqrt{2\frac{n(n-1)}{2}+\frac{1}{4}}
\]
\[
    \chi (G)\leq \frac{1}{2} + \sqrt{n^2-n+\frac{1}{4}}
\]
\[
    \chi (G)\leq \frac{1}{2} + n-\frac{1}{2}
\]
\[
    \chi (G)\leq n
\]
Вече доказахме, че $\chi (G)\leq n$ е изпълнено, следователно:
\[
    \chi (G)\leq \frac{1}{2} + \sqrt{2m+\frac{1}{4}}
\]
също е вярно.

\section*{Задача 3}
\begin{center}
    \includegraphics*[scale=0.2]{img1}
\end{center}

Нека обозначим левия граф $G_1(V_1,E_1)$ и десния $G_2(V_2,E_2)$. Можем да направим\\
няколко наблюдения:
\begin{itemize}
    \item $|V(G_1)|=|V(G_2)|=12$
    \item $|E(G_1)|=|E(G_2)|=18$
    \item $G_1$ и $G_2$ са 3-регулярни
\end{itemize}
Ако започнем елементарно пренареждане на върховете на $G_2$, така че да бъде
видимо изоморфен на $G_1$, ще съставим 2 нови индуцирани по върхове графа - $G_1'(V_1',E_1')$
и $G_2'(V_2',E_2')$, като $V_1'$ и $V_2'$ ще бъдат съответно върховете от циклите
с дължина 3, ясно изразен и в двата графа:
\begin{center}
    \includegraphics*[scale=0.2]{img2}
\end{center}
\newpage
Ето тук се натъкваме на проблем. Нито един от 3-те върха в $V_1'$
не е съседен с връх, който да е част от цикъл с размер 4,
в който не участва връх от $V_2'$, както е един от съседните върхове на връх от $V_2'$.
\begin{center}
    \includegraphics*[scale=0.2]{img3}
\end{center}
С това можем да заключим, че е невъзможно да пренаредим $G_2$, така че да
заеме формата на $G_1$, с което можем да заключим, че $G_1$ и $G_2$ не са изоморфни.

\section*{Задача 4}
\[
    \text{Редица S е дървесна} \Longleftrightarrow \sum_{i=1}^{n}d(i)=2n-2
\]
\begin{itemize}
    \item[($\Rightarrow)$]$\left|\begin{array}{c}
            \forall G(V,E):\sum_{u\in V}d(u)=2m \\
            \text{За всяко дърво: }m=n-1
        \end{array}\right.\Longrightarrow\sum_{i=1}^{n}d(i)=2(n-1)=2n-2$
    \item[($\Leftarrow)$]Използваме индукция по n (броя на върховете в дърво):\\
    \underline{База} n=2, $\sum_{i=1}^{n}d(i)=2n-2$:\\ В дърво с брой на върховете 2, всички степени на върховете са равни на 1. \checkmark\\
    \underline{Индукционно допускане} $\sum_{i=1}^{k}d(i)=2k-2$ е в сила за $\forall k<n$\\
    \underline{Индукционнa стъпка} Доказваме за дърво $T$ с n-върха.\\
    Взимаме произволен връх $e$, което не е свързан с връх от степен 1.
    Ако изтрием връх $e$, получаваме две нови дървета: $T_1=(T-e)_1$ и $T_2=(T-e)_2$
    и за тях знаем, че:
    \[
        n=|V(T_1)|+|V(T_2)|
    \]
    и $|V(T_1)|$ и $|V(T_2)|$ са по-малки от n. Тогава чрез индукционната стъпка:
    \[\begin{array}{c}
            \text{За }T_1: \sum_{i=1}^{n_1}d(i)=2n_1-2: \\\\
            \text{За }T_2: \sum_{i=1}^{n_2}d(i)=2n_2-2:
        \end{array}
    \]
    Когато върнем върха $e$, при свързването на двете дървета се получават две нови ребра,
    по едно инцидентно с $e$ във всяко дърво. Следователно:
    \[
        \sum_{i=1}^{n}d(i)=(2n_1-2+1)-(2n_2-2+1)=2(n_1+n_2)+2=2n+2
    \]


\end{itemize}

\end{document}