\documentclass{article}

\usepackage{amsmath} 
\usepackage[utf8]{inputenc}
\usepackage[T2A]{fontenc}
\usepackage[bulgarian,english]{babel}
\usepackage{amssymb}
\title{Формули и дефиниции}
\author{Александър Гуров}
\date{\datebulgarian{\today}}

\begin{document}
\maketitle
\section{Обикновени графи}

\begin{itemize}
    \item[] \underline{\textbf{Def:}} Граф $G=(V,E), E\subseteq\{X\subseteq V: |X|=2\} $

    \item[] Празен граф: $E=\varnothing $

    \item[] Тривиален граф: $|V|=1$

    \item[] \underline{\textbf{Def:}} Ако $\exists e=(u,v)\in E$, то u и v са \underline{съседи}

    \item[] Ако $e_1\cap e_2 = \varnothing$, то $e_1$ и $e_2$ са \underline{инцидентни}

    \item[] \underline{\textbf{Def:}}$N(u)=\{ v\in V | u \text{ и } v \text{ са съседи}\}$

    \item[] \underline{\textbf{Def:}}$N[u]=N(u)\cup \{u\}$

    \item[] \underline{\textbf{Def:}}$\mathcal{J}(u)=\{ e\in E | e \text{ е инцидентно с } u \}$

    \item[] Степен на връх: $d(u)= |\mathcal{J}(u)|$

    \item[] $\Delta (G)= max\{d(u) |  u\in V\}$

    \item[] $\delta(G)= min\{d(u) |  u\in V\}$

    \item[] \underline{\textbf{Максимална степен на връх}} $\Delta(u)\leq n-1$

    \item[] \underline{\textbf{Лема 1:}} $\sum_{u\in V}d(u)=2m$

    \item[] \underline{\textbf{Лема 2}} - \emph{the handshake lemma}: \\ $\forall G(V,E) \text{ с поне 2 върха, } \exists u,v\in V, u \neq v: d(u)=d(v)$

    \item[] G е \textbf{k-регулярен}, ако $\forall u\in V: d(u)=k$

    \item[] G е \textbf{пълен граф}, ако има всички възможни ребра при даденото множеството върхове

    \item[] $K_n$ е пълен граф на n върха, и е (n-1)-регулярен

    \item[] $K_n$ има точно $\binom{n}{2}$ ребра

    \item[] Подграф на $G$, \emph{индуциран} от $E'$, e $G'(U, E')$, където $E'\subseteq E$,\\ $U=\{u\in V|\exists e \in E', \exists x \in V:e=(u,x)\}$

    \item[] Подграф на $G$, \emph{индуциран} от $U$, e $G'(U, E')$, където $U\subseteq V$,\\ $E'=\{e\in E|\exists u,v \in U:e=(u,v)\}$

    \item[] Клика е подмножество от $V$, между чиито всеки два върха има ребро. К-клика е клика с к-върха.

    \item[] Антилика е подмножество от $V$, между чиито никои два върха няма ребро

    \item[] $\omega (G)$ - Кликово число е мощността на максималната клика в $G$

    \item[] $\alpha (G)$ - Число на независимост е мощността на максималната \\ антиклика в $G$

    \item[] $\bar{G}(V,E')$ е допълнението на $G$, където $E'=\{(u,v)|u,v\in V, u\neq v\} \setminus E$

    \item[] $\omega (G) = \alpha (\bar{G})$

    \item[] \underline{\textbf{Теорема 1:}} Нека $G(V,E)$ има поне 6 върха. В $G$ има поне 3-клика или 3-антиклика
\end{itemize}

\section{Неориентирани мултиграфи}

\end{document}