\documentclass{article}

\usepackage{amsmath} 
\usepackage[utf8]{inputenc}
\usepackage[T2A]{fontenc}
\usepackage[bulgarian,english]{babel}
%\usepackage[ddmmyyyy]{datetime}
\usepackage{framed} 

\title{Домашно 2}
\author{Александър Гуров}
\date{\datebulgarian{\today}}

\begin{document}

\maketitle

\section*{Задача 1} %1
От уравненията:
\[
    a_1 = (1, -1, 1, -1, -1),  a_2 = (2, 1, 1, 1, -1),  a_3 = (1, 1, -1, 1, 1)
\]
Съставяме система уравнения:
\[
    \left| \begin{array}{cccccc}
        x_1   & - x_2 & + x_3 & - x_4 & - x_5 & =0 \\
        2 x_1 & + x_2 & + x_3 & + x_4 & - x_5 & =0 \\
        x_1   & + x_2 & - x_3 & + x_4 & + x_5 & =0
    \end{array}
    \right.
\]
A от нея съставяме матрица:
\[
    \left(
    \begin{array}{ccccc}
            1 & -1 & 1  & -1 & -1 \\
            2 & 1  & 1  & 1  & -1 \\
            1 & 1  & -1 & 1  & 1
        \end{array}
    \right)
\]
И я решаваме:
\[
    \sim\left(
    \begin{array}{ccccc}
            1 & -1 & 1  & -1 & -1 \\
            0 & 3  & -1 & 3  & 1  \\
            0 & 2  & -2 & 2  & 2
        \end{array}
    \right)
    \sim\left(
    \begin{array}{ccccc}
            1 & -1 & 1            & -1 & -1           \\
            0 & 3  & -1           & 3  & 1            \\
            0 & 0  & -\frac{4}{3} & 0  & -\frac{4}{3}
        \end{array}
    \right)
\]
\[
    \sim\left(
    \begin{array}{ccccc}
            1 & -1 & 1  & -1 & -1 \\
            0 & 3  & -1 & 3  & 1  \\
            0 & 0  & 1  & 0  & -1
        \end{array}
    \right)
    \sim\left(
    \begin{array}{ccccc}
            1 & -1 & 0 & -1 & 0  \\
            0 & 3  & 0 & 3  & 0  \\
            0 & 0  & 1 & 0  & -1
        \end{array}
    \right)
\]
\begin{equation}
    \label{matrixU}
    \sim\left(
    \begin{array}{ccccc}
            1 & 0 & 0 & 0 & 0  \\
            0 & 1 & 0 & 1 & 0  \\
            0 & 0 & 1 & 0 & -1
        \end{array}
    \right)
\end{equation}
Векторите $a_1 = (1,0,0,0,0), a_2 = (0,1,0,1,0)$ и $a_3=(0,0,1,0,-1)$ са ЛНЗ и образуват базис базис на U.
\newpage
Дадена е хомогенната система линейни уравнения за U:
\[
    \left| \begin{array}{cccccc}
        x_1   & - x_2 & + 2 x _3 & - 3 x_4 & + 7 x_5 & =0 \\
        2 x_1 & + x_2 & - 3 x_3  & - 6 x_4 & + 4 x_5 & =0 \\
        x_1   & - x_2 & - 2 x_3  & - 3 x_4 & + 3 x_5 & =0
    \end{array}
    \right.
\]
От тази система линейни уравнения съставяме матрица и я решаваме:
\[
    \left( \begin{array}{ccccc}
            1 & -1 & 2  & -3 & 7 \\
            2 & 1  & -3 & -6 & 4 \\
            1 & -1 & -2 & -3 & 3
        \end{array}
    \right)
\]
\[
    \sim\left(\begin{array}{ccccc}
            1 & -1 & 2  & -3 & 7   \\
            0 & 3  & -7 & 0  & -10 \\
            0 & 0  & -4 & 0  & -4
        \end{array}
    \right)
    \sim\left(\begin{array}{ccccc}
            1 & -1 & 0 & -3 & 5  \\
            0 & 3  & 0 & 0  & -3 \\
            0 & 0  & 1 & 0  & 1
        \end{array}
    \right)
\]
\begin{equation}\label{oprostenaSistemaW}
    \sim\left(\begin{array}{ccccc}
            1 & 0 & 0 & -3 & 4  \\
            0 & 1 & 0 & 0  & -1 \\
            0 & 0 & 1 & 0  & 1
        \end{array}
    \right)
\end{equation}
Съставяме ФСР с независими променливи $x_4$ и $x_5$:
\begin{center}
    $x_4=p, x_5=q: x_3=-q, x_2=q, x_1=3p-4q$ \\
    $(3p-4q,q,-q,p,q)$\\
    При $x_4=1:(3,0,0,1,0)$\\
    При $x_5=1:(-4,1,-1,0,1)$\\
\end{center}
\begin{align}\label{fsrW}
    \text{ФСР на W:}
    \left|\begin{array}{ccccc}
              3x_1  & +x_4 & =0               \\
              -4x_1 & +x_2 & -x_3 & +x_5 & =0
          \end{array}\right.
\end{align}
\begin{align}\label{bazisW}
    \text{Базис на W:}
    \left(\begin{array}{ccccc}
                  3  & 0 & 0  & 1 & 0 \\
                  -4 & 1 & -1 & 0 & 1
              \end{array}\right)
\end{align}

За да намерим базиса на $U+W$, трябва да обединим техните базиси (на $U$(\ref*{matrixU}), на W(\ref*{bazisW})):
\[
    \left( \begin{array}{ccccc}
            1  & -1 & 1  & -1 & -1 \\
            2  & 1  & 1  & 1  & -1 \\
            1  & 1  & -1 & 1  & 1  \\
            3  & 0  & 0  & 1  & 0  \\
            -4 & 1  & -1 & 0  & 1
        \end{array}
    \right)
\]
\[
    \sim\left( \begin{array}{ccccc}
            1 & -1 & 1  & -1 & -1 \\
            0 & 3  & -1 & 3  & 1  \\
            0 & 2  & -2 & 2  & 2  \\
            0 & 3  & -3 & 4  & 3  \\
            0 & -3 & 3  & -4 & -3
        \end{array}
    \right)
    \sim\left( \begin{array}{ccccc}
            1 & -1 & 1  & -1 & -1 \\
            0 & 3  & -1 & 3  & 1  \\
            0 & 1  & -1 & 1  & 1  \\
            0 & 3  & -3 & 4  & 3
        \end{array}
    \right)
\]
\[
    \sim\left( \begin{array}{ccccc}
            1 & 0 & 0  & 0 & 0  \\
            0 & 0 & 2  & 0 & -2 \\
            0 & 1 & -1 & 1 & 1  \\
            0 & 0 & 0  & 1 & 0
        \end{array}
    \right)
    \sim\left( \begin{array}{ccccc}
            1 & 0 & 0 & 0 & 0  \\
            0 & 0 & 1 & 0 & -1 \\
            0 & 1 & 0 & 1 & 0  \\
            0 & 0 & 0 & 1 & 0
        \end{array}
    \right)
\]
Получаваме базиса на $U+W$. Сега чрез твърдението:
\[
    dim(U+W)=dimU+dimW-dim(U \cap W)
\]
Намираме размерността на $dim(U \cap W)$:
\[
    4=3+2-dim(U \cap W)
\]
\[
    dim(U \cap W)=1
\]
За да намерим базиса на сечението $U \cap W$,
трябва да обединим ФСР на U и
дадената хомогенната система линейни уравнения за W.
От базиса на U(\ref*{matrixU}) съставяме следната ФСР на U:
\[
    \left(
    \begin{array}{ccccc}
            1 & 0 & 0 & 0 & 0  \\
            0 & 1 & 0 & 1 & 0  \\
            0 & 0 & 1 & 0 & -1
        \end{array}
    \right)
\]
\[
    x_1 = 0 \ x_2=q, \ x_3=s, \ x_4=-q, \ x_5=s
\]
\begin{center}
    При $x_2 = 1: (0, 1, 0, -1, 0)$ \\
    При $x_3 = 1: (0, 0, 1, 0, 1)$
\end{center}
\begin{align}\label{fsrU}
    \text{ФСР на U:}
    \left|\begin{array}{ccccc}
              x_2 & -x_4 & =0 \\
              x_3 & x_5  & =0
          \end{array}\right.
\end{align}
Обединяваме ФСР на U(\ref*{fsrU}) и дадената в условието система за W:
\[
    \left| \begin{array}{cccccc}
        x_1   & - x_2 & + 2 x_3 & - 3 x_4 & + 7 x_5 & =0 \\
        2 x_1 & + x_2 & - 3 x_3 & - 6 x_4 & + 4 x_5 & =0 \\
        x_1   & - x_2 & - 2 x_3 & - 3 x_4 & + 3 x_5 & =0 \\
        x_2   & - x_4 & =0                               \\
        x_3   & + x_5 & =0
    \end{array}
    \right.
\]
Съставяме матрица и я решаваме:
\[
    \left(\begin{array}{ccccc}
            1 & -1 & 2  & -3  & 7 \\
            2 & 1  & -3 & -6  & 4 \\
            1 & -1 & -2 & -3  & 3 \\
            0 & 1  & 0  & - 1 & 0 \\
            0 & 0  & 1  & 0   & 1
        \end{array}
    \right)
\]
Заместваме първите 3 реда с вече опростената матрица на системата\\ уравнения за W(\ref*{oprostenaSistemaW}):
\[
    \sim\left( \begin{array}{ccccc}
            1 & 0 & 0 & -3  & 4  \\
            0 & 1 & 0 & 0   & -1 \\
            0 & 0 & 1 & 0   & 1  \\
            0 & 1 & 0 & - 1 & 0  \\
            0 & 0 & 1 & 0   & 1
        \end{array}
    \right)
    \sim\left( \begin{array}{ccccc}
            1 & 0 & 0 & -3  & 4  \\
            0 & 1 & 0 & 0   & -1 \\
            0 & 1 & 0 & - 1 & 0  \\
            0 & 0 & 1 & 0   & 1
        \end{array}
    \right)
\]
\newpage
За да намерим базис на $U \cap W$, трябва да направим ФСР на $U \cap W$.
Съставяме ФСР с независима променлива $x_5$:
\[
    x_1 = -p, \ x_2=p, \ x_3=-p, \ x_4=p, \ x_5=p
\]
\begin{center}
    При $x_5 = 1: (-1, 1, -1, 1, 1)$ \\
    ФСР на $U \cap W :-x_1 +x_2 - x_3 +x_4 +x_5 =0$\\
    Базис на $U \cap W:(-1, 1, -1, 1, 1)$
\end{center}
\end{document}