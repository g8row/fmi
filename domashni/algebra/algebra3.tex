\documentclass{article}

\usepackage{amsmath} 
\usepackage[utf8]{inputenc}
\usepackage[T2A]{fontenc}
\usepackage[bulgarian,english]{babel}
%\usepackage[ddmmyyyy]{datetime}
\usepackage{framed} 

\title{Домашно 2}
\author{Александър Гуров}
\date{\datebulgarian{\today}}

\begin{document}

\maketitle
\section*{Задача 3}

От дефинициите на линейните изображения съставяме матрици:
\[
    \theta : U \longrightarrow V,
\]
\[
    \theta ( x_1 e_1 + x_2 e_2 + x_3 e_3) = ( 2 x_1 + 3 x_2 - x_3) f_1 + ( x_1 - 2 x_2 - x_3) f_2, \ \ \forall x_1, x_2, x_3 \in F,
\]
\[
    \text{Матрица на $\theta$: }
    A=
    \left(\begin{array}{ccc}
            2 & 3  & -1 \\
            1 & -2 & -1
        \end{array}
    \right)
\]

\[
    \varphi : U \longrightarrow U,
\]
\begin{align*}
    \varphi ( x_1 e_1 + x_2 e_2 + x_3 e_3) = ( x _1 - x_2 + 2 x_3) e_1 + ( 3 x_1 + x_2 - x_3) e_2 + & ( 2 x_1 + x_2 + x_3) e_3,   \\
                                                                                                    & \forall x_1, x_2, x_3 \in F
\end{align*}
\[
    \text{Матрица на $\varphi$: }
    B=
    \left(\begin{array}{ccc}
            1 & -1 & 2  \\
            3 & 1  & -1 \\
            2 & 1  & 1
        \end{array}
    \right)
\]

\[
    \psi : V \longrightarrow V,
\]
\[
    \psi ( y_1 f_1 + y_2 f_2) = ( 5 y_1 - y_2) f_1 + ( 2 y_1 - 4 y_2) f_2, \ \ \forall y_1, y_2 \in F
\]
\[
    \text{Матрица на $\psi$: }
    C=
    \left(\begin{array}{cc}
            5 & -1 \\
            2 & -4
        \end{array}
    \right)
\]
\newpage
\[
    \text{Търсим векторите $u\in U$, за които е вярно}( \psi \theta - \theta \varphi) (u) = f_1
\]
\[
    u\in U \implies u = x.e = (x_1e_1,x_2e_2) = x_1e_1+x_2e_2
\]
\begin{center}
    $( \psi \theta - \theta \varphi) (u) =\theta(\psi(u)) -  \varphi(\theta(u)) = \theta(\psi(ex)) -  \varphi(\theta(ex))$
    $= (f.A_{\psi\theta}-f.A_{\theta\varphi}).x = f.x.(A_\psi A_\theta-A_\theta A_\varphi)=$\\
    $=f.x.(CA-AB)$
\end{center}
\begin{center}
    $( \psi \theta - \theta \varphi) (u) = f_1$\\
    $f.x.(CA-AB) = f\left(\begin{array}{c}
                1 \\
                0
            \end{array}\right)$\\
    $x=\left(\begin{array}{c}
            1 \\
            0
        \end{array}\right)(CA-AB)^{-1}$
\end{center}
\[
    CA=
    \left(\begin{array}{cc}
            5 & -1 \\
            2 & -4
        \end{array}
    \right)\left(\begin{array}{ccc}
            2 & 3  & -1 \\
            1 & -2 & -1
        \end{array}
    \right)\sim
    \left(\begin{array}{ccc}
            9 & 17 & -4 \\
            0 & 14 & 2
        \end{array}
    \right)
\]
\[
    AB=
    \left(\begin{array}{ccc}
            2 & 3  & -1 \\
            1 & -2 & -1
        \end{array}
    \right)
    \left(\begin{array}{ccc}
            1 & -1 & 2  \\
            3 & 1  & -1 \\
            2 & 1  & 1
        \end{array}
    \right)\sim
    \left(\begin{array}{ccc}
            9  & 0  & 0 \\
            -7 & -4 & 3
        \end{array}
    \right)
\]
\[
    CA-AB=
    \left(\begin{array}{ccc}
            0 & 17 & -4 \\
            7 & 10 & -1
        \end{array}
    \right)
\]
\[
    (CA-AB)^{-1}=
    \left(\begin{array}{ccc}
            7 & 10 & -1 \\
            0 & 17 & -4 \\
            0 & 0  & 1
        \end{array}\left|
    \begin{array}{ccc}
            1 & 0 & 0 \\
            0 & 1 & 0 \\
            0 & 0 & 1
        \end{array}\right.
    \right)
    \sim
    \left(\begin{array}{ccc}
            7 & 10 & -1 \\
            0 & 17 & 0  \\
            0 & 0  & 1
        \end{array}\left|
    \begin{array}{ccc}
            1 & 0 & 0 \\
            0 & 1 & 4 \\
            0 & 0 & 1
        \end{array}\right.
    \right)
\]
\[
    \sim
    \left(\begin{array}{ccc}
            7 & 10 & -1 \\
            0 & 1  & 0  \\
            0 & 0  & 1
        \end{array}\left|
    \begin{array}{ccc}
            1 & 0            & 0            \\
            0 & \frac{1}{17} & \frac{4}{17} \\
            0 & 0            & 1
        \end{array}\right.
    \right)
    \sim
    \left(\begin{array}{ccc}
            7 & 0 & -1 \\
            0 & 1 & 0  \\
            0 & 0 & 1
        \end{array}\left|
    \begin{array}{ccc}
            1 & \frac{-10}{17} & \frac{-40}{17} \\
            0 & \frac{1}{17}   & \frac{4}{17}   \\
            0 & 0              & 1
        \end{array}\right.
    \right)
\]
\[
    \sim
    \left(\begin{array}{ccc}
            7 & 0 & 0 \\
            0 & 1 & 0 \\
            0 & 0 & 1
        \end{array}\left|
    \begin{array}{ccc}
            1 & \frac{-10}{17} & \frac{-23}{17} \\
            0 & \frac{1}{17}   & \frac{4}{17}   \\
            0 & 0              & 1
        \end{array}\right.
    \right)
    \sim
    \left(\begin{array}{ccc}
            1 & 0 & 0 \\
            0 & 1 & 0 \\
            0 & 0 & 1
        \end{array}\left|
    \begin{array}{ccc}
            \frac{1}{7} & \frac{-10}{119} & \frac{-23}{119} \\
            0           & \frac{1}{17}    & \frac{4}{17}    \\
            0           & 0               & 1
        \end{array}\right.
    \right)
\]
\[
    x=\left(\begin{array}{ccc}
            1 \\
            0 \\
        \end{array}\right)
    \left(\begin{array}{ccc}
            \frac{1}{7} & \frac{-10}{119} & \frac{-23}{119} \\
            0           & \frac{1}{17}    & \frac{4}{17}    \\
            0           & 0               & 1
        \end{array}\right)
\]
\end{document}
