\documentclass{article}

\usepackage{amsmath} 
\usepackage[utf8]{inputenc}
\usepackage[T2A]{fontenc}
\usepackage[bulgarian,english]{babel}
%\usepackage[ddmmyyyy]{datetime}
\usepackage{framed} 

\title{Домашно 2}
\author{Александър Гуров}
\date{\datebulgarian{\today}}

\begin{document}

\maketitle
\section*{Задача 2}
\begin{align*}
    \Phi ( x_1 e_1 + x_2 e _2 + x_3 e_3 + x_4 e_4) = & ( 4 x_1 - 3 x_2 + x_3 + 2 x_4) e_1 + ( x_1 - x_2 + x_3 + x_4) e_2 \\
                                                     & + ( x_1 - 2 x_3 + q x_4) e_3 + ( p x_1 + x_2 -5 x_3 - 3 x_4) e_4
\end{align*}

Съставяме матрица спрямо $(e_1,e_2,e_3,e_4)$:
\[
    \left(\begin{array}{cccc}
            4 & -3 & 1  & 2  \\
            1 & -1 & 1  & 1  \\
            1 & 0  & -2 & q  \\
            p & 1  & -5 & -3
        \end{array}\right)
    \left(\begin{array}{c}
            v_1 \\
            v_2 \\
            v_3 \\
            v_4
        \end{array}\right)
\]
Намираме базиса на $ker(\Phi)$ чрез векторите с координати $(x_1,x_2,x_3)$ спрямо базиса $(e_1,e_2,e_3)$,
които са от $ker(\Phi)$, тоест отиват в $\vec{0}$.
\[
    \left(\begin{array}{cccc}
            4 & -3 & 1  & 2  \\
            1 & -1 & 1  & 1  \\
            1 & 0  & -2 & q  \\
            p & 1  & -5 & -3
        \end{array}\right)
    \left(\begin{array}{c}
            v_1 \\
            v_2 \\
            v_3 \\
            v_4
        \end{array}\right)
    =0
\]
\[
    \left(\begin{array}{cccc}
            1 & -1 & 1  & 1  \\
            4 & -3 & 1  & 2  \\
            1 & 0  & -2 & q  \\
            p & 1  & -5 & -3
        \end{array}\right)
    \sim\left(\begin{array}{cccc}
            1 & -1  & 1    & 1    \\
            0 & 1   & -3   & -2   \\
            0 & 1   & -3   & q-1  \\
            0 & 1+p & -p-5 & -p-3
        \end{array}\right)
\]
\[
    \sim\left(\begin{array}{cccc}
            1 & -1 & 1    & 1   \\
            0 & 1  & -3   & -2  \\
            0 & 0  & 0    & q+1 \\
            0 & 0  & 2p-2 & p-1
        \end{array}\right)
    \sim\left(\begin{array}{cccc}
            1 & -1 & 1    & 1   \\
            0 & 1  & -3   & -2  \\
            0 & 0  & 2p-2 & p-1 \\
            0 & 0  & 0    & q+1
        \end{array}\right)
\]
\newpage
\textbf{I-ви случай:} $q=-1, p=1$
\[
    \left(\begin{array}{cccc}
            1 & -1 & 1  & 1  \\
            0 & 1  & -3 & -2 \\
            0 & 0  & 0  & 0  \\
            0 & 0  & 0  & 0
        \end{array}\right)
    \sim\left(\begin{array}{cccc}
            1 & 0 & -2 & -1 \\
            0 & 1 & -3 & -2 \\
            0 & 0 & 0  & 0  \\
            0 & 0 & 0  & 0
        \end{array}\right)
\]
Съставяме ФСР с независими променливи $e_3=m, e_4=n$:
\[
    e_1=2m+n, e_2=3m+2n, e_3=m, e_4=n
\]
\begin{center}
    При $m=1:(2,3,1,0)$\\
    При $n=1:(1,2,0,1)$
\end{center}
\begin{align*}
    \text{Базис на $ker(\Phi):$}
     & \left(\begin{array}{cccc}
                     2 & 3 & 1 & 0 \\
                     1 & 2 & 0 & 1
                 \end{array}
    \right)
\end{align*}
\[
    dim(ker(\Phi))+dim(im(\Phi))=m
\]
Чрез водещите единици на базиса на $ker(\Phi)$ съставяме базиса на $im(\Phi)$.
\begin{align*}
    \text{Базис на $im(\Phi):$}
     & \left(\begin{array}{cccc}
                     4  & 1  & 1 & 1 \\
                     -3 & -1 & 0 & 1
                 \end{array}
    \right)
\end{align*}

\textbf{II-ри случай:} $q\neq-1, p=1$
\[
    \left(\begin{array}{cccc}
            1 & -1 & 1  & 1   \\
            0 & 1  & -3 & -2  \\
            0 & 0  & 0  & 0   \\
            0 & 0  & 0  & q+1
        \end{array}\right)
    \sim\left(\begin{array}{cccc}
            1 & 0 & -2 & -1 \\
            0 & 1 & -3 & -2 \\
            0 & 0 & 0  & 0  \\
            0 & 0 & 0  & 1
        \end{array}\right)
    \sim\left(\begin{array}{cccc}
            1 & 0 & -2 & 0 \\
            0 & 1 & -3 & 0 \\
            0 & 0 & 0  & 0 \\
            0 & 0 & 0  & 1
        \end{array}\right)
\]
Съставяме ФСР с независимa променливa $e_3=m$:
\[
    e_1=2m, e_2=3m, e_3=m, e_4=0
\]
\begin{center}
    При $m=1:(2,3,1,0)$\\
    Базис на $ker(\Phi):(2,3,1,0)$
\end{center}
\begin{align*}
    \text{Базис на $im(\Phi):$}
     & \left(\begin{array}{cccc}
                     4  & 1  & 1 & 1  \\
                     -3 & -1 & 0 & 1  \\
                     2  & 1  & q & -3
                 \end{array}
    \right)
\end{align*}

\textbf{III-ти случай:} $q=-1, p\neq1$
\[
    \left(\begin{array}{cccc}
            1 & -1 & 1    & 1   \\
            0 & 1  & -3   & -2  \\
            0 & 0  & 2p-2 & p-1 \\
            0 & 0  & 0    & 0
        \end{array}\right)
    \sim\left(\begin{array}{cccc}
            1 & 0 & -2 & -1 \\
            0 & 1 & -3 & -2 \\
            0 & 0 & 2  & 1  \\
            0 & 0 & 0  & 0
        \end{array}\right)
    \sim\left(\begin{array}{cccc}
            1 & 0 & 0 & 0 \\
            0 & 1 & 1 & 0 \\
            0 & 0 & 2 & 1 \\
            0 & 0 & 0 & 0
        \end{array}\right)
\]
Съставяме ФСР с независимa променливa $e_3=m$:
\[
    e_1=0, e_2=-m, e_3=m, e_4=-2m
\]
\begin{center}
    При $m=1:(0,-1,1,-2)$\\
    Базис на $ker(\Phi):(0,-1,1,-2)$
\end{center}
\begin{align*}
    \text{Базис на $im(\Phi):$}
     & \left(\begin{array}{cccc}
                     4  & 1  & 1  & p  \\
                     -3 & -1 & 0  & 1  \\
                     1  & 1  & -2 & -5
                 \end{array}
    \right)
\end{align*}

\textbf{IV-ти случай:} $q\neq-1, p\neq1$
\[
    \left(\begin{array}{cccc}
            1 & -1 & 1    & 1   \\
            0 & 1  & -3   & -2  \\
            0 & 0  & 2p-2 & p-1 \\
            0 & 0  & 0    & q+1
        \end{array}\right)
    \sim\left(\begin{array}{cccc}
            1 & -1 & 1  & 1  \\
            0 & 1  & -3 & -2 \\
            0 & 0  & 2  & 1  \\
            0 & 0  & 0  & 1
        \end{array}\right)
    \sim\left(\begin{array}{cccc}
            1 & 0 & 0 & 0 \\
            0 & 1 & 0 & 0 \\
            0 & 0 & 1 & 0 \\
            0 & 0 & 0 & 1
        \end{array}\right)
\]
\begin{center}
    Не съществува базис на $ker(\Phi)$.
\end{center}
\begin{align*}
    \text{Базис на $im(\Phi):$} &
    \left(\begin{array}{cccc}
                  4  & 1  & 1  & p  \\
                  -3 & -1 & 0  & 1  \\
                  1  & 1  & -2 & -5 \\
                  2  & 1  & q  & -3
              \end{array}\right)
\end{align*}


\end{document}