\documentclass{article}

\usepackage{amsmath} 
\usepackage[utf8]{inputenc}
\usepackage[T2A]{fontenc}
\usepackage[bulgarian,english]{babel}
%\usepackage[ddmmyyyy]{datetime}
\usepackage{framed} 

\title{Домашно 2}
\author{Александър Гуров}
\date{\datebulgarian{\today}}

\begin{document}

\maketitle
\section*{Задача 4}

Първо ще търсим $\lambda$, за което е вярно $det(A-\lambda E)=0$.
\begin{align*}
    A-\lambda E=
    \left(
    \begin{array}{ccc}
        9  & 2  & 14  \\
        16 & 5  & 28  \\
        -8 & -2 & -13
    \end{array}
    \right)
    -\lambda E=
    \left(\begin{array}{ccc}
              9-\lambda & 2         & 14          \\
              16        & 5-\lambda & 28          \\
              -8        & -2        & -13-\lambda
          \end{array}
    \right)
\end{align*}
\begin{align*}
    det(A-\lambda E)=
    \left|\begin{array}{ccc}
              9-\lambda & 2         & 14          \\
              16        & 5-\lambda & 28          \\
              -8        & -2        & -13-\lambda
          \end{array}
    \right|
\end{align*}
\begin{align*}
    =(9-\lambda)(5-\lambda)(-13-\lambda)+2.28  & .(-8)  +14.16.(-2)-(9-\lambda).28.(-2)                                \\
                                               & -14.(5-\lambda).(-8)-2.16.(-13-\lambda)=                              \\
    =(45-14\lambda+\lambda^2)(-13-\lambda)-44  & 8  -448+(504-56\lambda)+(560-112\lambda)                              \\
                                               & +(416+32\lambda)                                                      \\
    =-(585-182\lambda+13\lambda^2+45\lambda-14 & \lambda^2+\lambda^3)  + 584-136\lambda=-1+\lambda+\lambda^2-\lambda^3
\end{align*}
\[
    det(A-\lambda E)=0
\]
\[
    -1+\lambda+\lambda^2-\lambda^3=0
\]
\[
    \text{По метод на Хорнер: }
    \begin{array}{ccccc}
           & -1 & 1 & 1 & -1 \\
        1| & -1 & 0 & 1 & 0
    \end{array}
    \implies(\lambda -1)(\lambda ^2-1)=0
\]
\[
    (\lambda -1)^2(\lambda +1)=0 \implies \lambda_1=1, \lambda_2=-1
\]

\textbf{I-ви случай:} $\lambda=1$
\[
    \left(\begin{array}{ccc}
            9-\lambda & 2         & 14          \\
            16        & 5-\lambda & 28          \\
            -8        & -2        & -13-\lambda
        \end{array}
    \right)
    \sim\left(\begin{array}{ccc}
            9-1 & 2   & 14    \\
            16  & 5-1 & 28    \\
            -8  & -2  & -13-1
        \end{array}
    \right)
    \sim\left(\begin{array}{ccc}
            8  & 2  & 14  \\
            16 & 4  & 28  \\
            -8 & -2 & -14
        \end{array}
    \right)
\]
\[
    \sim\left(\begin{array}{ccc}
            8  & 2 & 14 \\
            16 & 4 & 28 \\
            0  & 0 & 0
        \end{array}
    \right)
    \sim\left(\begin{array}{ccc}
            8 & 2 & 14 \\
            0 & 0 & 0  \\
            0 & 0 & 0
        \end{array}
    \right)
    \sim\left(\begin{array}{ccc}
            4 & 1 & 7 \\
            0 & 0 & 0 \\
            0 & 0 & 0
        \end{array}
    \right)
\]
\newpage
Съставяме ФСР с независими променливи $x_1=m, x_2=n$:
\[
    x_1=m, x_2=n, x_3=\frac{-4m-n}{7}
\]
\[
    \text{При } m=1: (1, 0, \frac{-4}{7})
\]
\[
    \text{При } n=1: (0, 1, \frac{-1}{7})
\]

\textbf{II-ри случай:} $\lambda=-1$
\[
    \left(\begin{array}{ccc}
            9-\lambda & 2         & 14          \\
            16        & 5-\lambda & 28          \\
            -8        & -2        & -13-\lambda
        \end{array}
    \right)
    \sim\left(\begin{array}{ccc}
            9+1 & 2   & 14    \\
            16  & 5+1 & 28    \\
            -8  & -2  & -13+1
        \end{array}
    \right)
    \sim\left(\begin{array}{ccc}
            10 & 2  & 14  \\
            16 & 6  & 28  \\
            -8 & -2 & -12
        \end{array}
    \right)
\]
\[
    \sim\left(\begin{array}{ccc}
            5  & 1  & 7  \\
            8  & 3  & 14 \\
            -4 & -1 & -6
        \end{array}
    \right)
    \sim\left(\begin{array}{ccc}
            0  & -\frac{1}{4} & -\frac{1}{2} \\
            0  & 1            & 2            \\
            -4 & -1           & -6
        \end{array}
    \right)
    \sim\left(\begin{array}{ccc}
            -4 & -1 & -6 \\
            0  & 1  & 2  \\
            0  & -1 & -2
        \end{array}
    \right)
\]
\[
    \sim\left(\begin{array}{ccc}
            -4 & 0 & -4 \\
            0  & 1 & 2  \\
            0  & 0 & 0
        \end{array}
    \right)
    \sim\left(\begin{array}{ccc}
            1 & 0 & 1 \\
            0 & 1 & 2 \\
            0 & 0 & 0
        \end{array}
    \right)
\]
Съставяме ФСР с независимa променливa $x_3=m$:
\[
    x_1=-m, x_2=-2m, x_3=m
\]
\[
    \text{При } m=1: (-1, -2, 1)
\]
\begin{align*}
    \text{Базис от собствени вектори на }V=
    \left(\begin{array}{ccc}
                  7  & 0  & -4 \\
                  0  & 7  & -1 \\
                  -1 & -2 & 1
              \end{array}
    \right)
\end{align*}
\begin{align*}
    \text{Диагонална матрица на $V$: }D=
    \left(\begin{array}{ccc}
                  1 & 0 & 0  \\
                  0 & 1 & 0  \\
                  0 & 0 & -1
              \end{array}
    \right)
\end{align*}


\end{document}