\documentclass{article}

\usepackage{amsmath}
\usepackage[utf8]{inputenc}
\usepackage[T2A]{fontenc}
\usepackage[bulgarian,english]{babel}
\usepackage{amssymb}
\usepackage{tcolorbox}
\usepackage[a4paper, total={6in, 8in}]{geometry}

\newcommand*{\bR}{\mathbb{R}}

\title{Теоретично домашно}
\author{Александър Гуров}
\date{\datebulgarian{\today}}

\begin{document}
\maketitle

\section{Задача}

\textbf{а)}
Функцията $f: \bR \rightarrow \bR$ е ограничена отгоре, ако съществува $\lambda \in \bR$, за което е изпълнено условието:
$\exists\lambda \in \bR \ \forall x \in \bR : f(x)\leq\lambda$ и $\lambda$ се нарича горна граница (мажоранта). Функцията $f$ не е ограничена отгоре,
ако е изпълнено:
\begin{align*}
    \neg(\exists\lambda \in \bR \ \forall x \in \bR : f(x)\leq\lambda)       & \equiv \\
    \equiv\forall\lambda \in \bR \ \neg(\forall x \in \bR : f(x)\leq\lambda) & \equiv \\
    \equiv\forall\lambda \in \bR \ \exists x \in \bR : \neg(f(x)\leq\lambda) & \equiv \\
    \equiv\forall\lambda \in \bR \ \exists x \in \bR : f(x)>\lambda          &
\end{align*}
\textbf{б)}
Функцията $f: \bR \rightarrow \bR$ има най-голяма стойност, ако съществува $x_1 \in \bR$, за което е изпълнено условието:
$\exists x_1 \in\bR \ \forall x_2 \in \bR: f(x_1)\geq f(x_2)$ и $\lambda$ се нарича горна граница (мажоранта). Функцията $f$ няма
най-голяма стойност, ако е изпълнено:
\begin{align*}
    \neg(\exists x_1 \in\bR \ \forall x_2 \in \bR: f(x_1)\geq f(x_2))       & \equiv \\
    \equiv\forall x_1 \in\bR \ \neg(\forall x_2 \in \bR: f(x_1)\geq f(x_2)) & \equiv \\
    \equiv\forall x_1 \in\bR \ \exists x_2 \in \bR: \neg(f(x_1)\geq f(x_2)) & \equiv \\
    \equiv\forall x_1 \in\bR \ \exists x_2 \in \bR: f(x_1)< f(x_2)          &
\end{align*}
\textbf{в)} (Koши) Нека $f : D \rightarrow \bR, D \subset \bR$. $f$ е непрекъсната в $x_0 \in D$, ако за всяко $\varepsilon > 0$
съществува $\delta > 0$ такова, че ако $x \in D$ и $|x - x_0| < \delta$, то $|f (x) - f (x_0)| < \varepsilon$.
Функцията $f$ се нарича непрекъсната, ако $f$ е непрекъсната във всяка точка от дефиниционната си област,
тoест:
\[
    \forall x_0 \in D \ \forall \varepsilon > 0 \ \exists \delta > 0 \  \forall x \in D, |x - x_0| < \delta : \ |f (x) - f (x_0)| < \varepsilon
\]
Функцията $f$ не е непрекъсната, когато:
\begin{align*}
    \neg(\forall x_0 \in D \ \forall \varepsilon > 0 \ \exists \delta > 0 \  \forall x \in D, |x - x_0| < \delta : \ |f (x) - f (x_0)| < \varepsilon)          & \equiv \\
    \equiv\exists x_0 \in D \ \neg(\forall \varepsilon > 0 \ \exists \delta > 0 \  \forall x \in D, |x - x_0| < \delta : \ |f (x) - f (x_0)| < \varepsilon)    & \equiv \\
    \equiv\exists x_0 \in D \ \exists \varepsilon > 0 \ \neg(\exists \delta > 0 \  \forall x \in D, |x - x_0| < \delta : \ |f (x) - f (x_0)| < \varepsilon)    & \equiv \\
    \equiv\exists x_0 \in D \ \exists \varepsilon > 0 \ \forall \delta > 0 \  \neg(\forall x \in D, |x - x_0| < \delta : \ |f (x) - f (x_0)| < \varepsilon)    & \equiv \\
    \equiv\exists x_0 \in D \ \exists \varepsilon > 0 \ \forall \delta > 0 \  \exists x \in D,\neg( |x - x_0| < \delta : \ |f (x) - f (x_0)| < \varepsilon)    & \equiv \\
    \equiv\exists x_0 \in D \ \exists \varepsilon > 0 \ \forall \delta > 0 \  \exists x \in D, |x - x_0| \geq \delta : \ \neg(|f (x) - f (x_0)| < \varepsilon) & \equiv \\
    \equiv\exists x_0 \in D \ \exists \varepsilon > 0 \ \forall \delta > 0 \  \exists x \in D, |x - x_0| \geq \delta : \ |f (x) - f (x_0)| \geq \varepsilon    &
\end{align*}

(Хайне) Нека $f : D \rightarrow \bR, \ D \subset \bR$. Kaзваме, че
функцията f е непрекъсната в $x_0 \in D$, ако за всяка редица $\{x_n\}^\infty_{n=1} \subset D$ от стойности на аргумента,
която клони към $x_0$, съответната редица от функционални стойности ${f(x_n)}^\infty_{n=1}$ клони към $f(x_0)$. Функцията $f$
е непрекъсната, ако $f$ е непрекъсната във всяка точка от дефиниционната си област, тoест:
\[
    \forall x_0 \in D \ \forall \{x_n\}^\infty_{n=1} \subset D, \ x_n \xrightarrow[n\rightarrow\infty]{} x_0 : f (x_n) \xrightarrow[n\rightarrow\infty]{} f(x_0)
\]
Функцията $f$ не е непрекъсната, когато:
\begin{align*}
    \neg(\forall x_0 \in D \ \forall \{x_n\}^\infty_{n=1} \subset D, \ x_n \xrightarrow[n\rightarrow\infty]{} x_0 : f (x_n) \xrightarrow[n\rightarrow\infty]{} f(x_0))             & \equiv \\
    \equiv\exists x_0 \in D \ \neg(\forall \{x_n\}^\infty_{n=1} \subset D, \ x_n \xrightarrow[n\rightarrow\infty]{} x_0 : f (x_n) \xrightarrow[n\rightarrow\infty]{} f(x_0))       & \equiv \\
    \equiv\exists x_0 \in D \ \exists \{x_n\}^\infty_{n=1} \subset D, \ \neg(x_n \xrightarrow[n\rightarrow\infty]{} x_0 : f (x_n) \xrightarrow[n\rightarrow\infty]{} f(x_0))       & \equiv \\
    \equiv\exists x_0 \in D \ \exists \{x_n\}^\infty_{n=1} \subset D, \ \neg(x_n \xrightarrow[n\rightarrow\infty]{} x_0) : \neg(f (x_n) \xrightarrow[n\rightarrow\infty]{} f(x_0)) &
\end{align*}

\section[2]{Задача}
\textbf{a)}  Числовата редица $\{a_n\}^\infty_{n=1}$ е ограничена, следователно $\exists b,c \in \bR \ \forall n \in \mathbb{N}:b\leq a_n \leq c$
По теоремата на Болцано-Вайерщрас (Принцип за компактност), в числовата редица същесвува поне една точка на сгъстяване.
Дефиницията на $lim \ sup \ a_n$ е най-дясната точка на сгъстяване на редицата. Следователно $\forall \varepsilon>0 \ \forall n \in \mathbb{N} : |a_n - lim \ sup \ a_n| < \varepsilon$.
Mножеството $V=\{\forall n\in \mathbb{N} \  \forall \varepsilon > 0: lim \ sup \ a_n + \varepsilon < a_n \leq c\}, V \subseteq\mathbb{N}$
от индекси на елементи от числовата редица, които се намират между $lim \ sup \ a_n$ и горната граница на числовата редица $\{a_n\}^\infty_{n=1}$
ще бъде крайно или празно. Тоест допълнението на $V$, $\overline{V}$ ще бъде кофинитно и ще бъде дефинирано
$\overline{V} = \{\forall n \in \mathbb{N} \ \forall \varepsilon > 0: a_n < lim  \ sup \ a_n +\varepsilon\}$,
с което доказваме, че $lim \ sup \ a_n$ е същесвена мажоранта. Множеството от същесвените мажоранти съвпада с множеството V,
от което е очевидно, че е ограничено отдолу и долната му граница съвпада с $lim \ sup \ a_n$.

\textbf{б)}
Можем да представим редицата $c_n$, като $sup(\{a_k\}^\infty_{k=1}\setminus\{a_k\}^n_{k=1})$. Тогава
\[
    \lim_{n\rightarrow 0} sup(\{a_k\}^\infty_{k=1}\setminus\{a_k\}^n_{k=1}) = sup(\{a_k\}^\infty_{k=1}) = lim \ sup \ a_n
\]

\textbf{в)} Както доказахме в 2a), $\overline{V}$ е кофинитно и $\overline{V} = \{\forall n \in \mathbb{N} \ \forall \varepsilon > 0: a_n < lim  \ sup \ a_n +\varepsilon\}$,
следователно почти всички членове на редицата се намират в $(-\infty, lim \ sup \ a_n + \varepsilon)$.

\textbf{г)}
От точките на сгъстяване на $\{a_n\}^\infty_{n=1}$ и $\{b_n\}^\infty_{n=1}$:
\begin{gather*}
    \forall\varepsilon > 0 \ \forall n_1 \in \mathbb{N}, \ \forall n\geq n_0: a_n \leq lim \ sup \ a_n + \frac{\varepsilon}{2}\\
    \forall\varepsilon > 0 \ \forall n_2 \in \mathbb{N}, \ \forall n\geq n_0: b_n \leq lim \ sup \ b_n + \frac{\varepsilon}{2}
\end{gather*}
При $n_0 = max(n_1,n_2)$. Следователно
\[
    \forall\varepsilon > 0 \ \forall n \geq n_0 : a_n + b_n \leq lim \ sup \ a_n + lim \ sup \ b_n + \varepsilon
\]
Тогава $lim \ sup \ _{n\rightarrow 0} \ (a_n + b_n) \leq lim \ sup \ _{n\rightarrow 0} \ a_n + lim \ sup \ _{n\rightarrow 0} \ b_n$.

\section[3]{Задача}
Функцията $f: (a, +\infty) \rightarrow \bR$ e диференцируема и от това знаем, че е непрекъсната.
Следователно за $\forall x \in (a, +\infty)$ съществува $f'(x)$. Също знаем, че $f$ е ограничена, тоест $\forall x \in (a, +\infty) \ \exists b, c \in \bR: b \leq f(x) \leq c$.
Нека вземем две произволни точки $m,n \in (a, +\infty), n>m$. От теоремата на Лагранж знаем, че съществува $\xi \in (m,n)$, за което е изпълнено
\[
    f'(\xi)=\frac{f(n)-f(m)}{n-m}
\]
Нека съставим две редици $\forall \varepsilon >0 \ \{m_k\}^\infty_{k=1}, m_k=a+k$, $\forall \varepsilon >0 \ \{n_k\}^\infty_{k=1}, n_k=a+k+1$.
Нека разгледаме производната на $x_k \in (m_k, n_k), k \rightarrow \infty$:
\[
    f'(x_k)=\frac{f(n_k)-f(m_k)}{n_k-m_k}
\]
Със сигурност можем да съставим $\{x_k\}^\infty_{k=1}$, които имат производна, само трябва да докажем, че $f'(x)=0$.
$f$ е ограничена, следователно $b\leq f(m_k)\leq c$ и $b\leq f(n_k)\leq c$ и
\[
    2b\leq \frac{f(n_k)-f(m_k)}{n_k-m_k} \leq 2c
\]
От редиците $n_k-m_k=a+k+1-a-k=1$:
\[
    2b\leq f(n_k)-f(m_k) \leq 2c
\]
ако $d=max(b,c)$, то
\[
    |f(n_k)-f(m_k)| \leq 2d
\]
Тъй като $f$ е ограничена, разликата на $f(n_k)$ и $f(m_k)$ ще намалява и ще се изпълни сходимостта във формата на Коши:
\[
    |f(n_k)-f(m_k) - 0| \leq \varepsilon \Rightarrow f(n_k)-f(m_k) \xrightarrow[k \rightarrow \infty]{} 0
\]
\[
    f'(x_k)\xrightarrow[k \rightarrow \infty]{} 0
\]
\end{document}